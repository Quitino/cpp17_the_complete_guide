\chapter{译者序}
这是我翻译的第一本英文资料。
去年刚读完Nicolai M. Josuttis的《C++标准库》(第二版)中文版之后,
感觉意犹未尽,想再寻找一些有关最新C++标准的书籍,机缘巧合之下发现了本书。

当时之所以会想到翻译这本书,主要有三个原因:
\begin{itemize}
    \item 一是太过无聊,无所事事。
    \item 二是我阅读英文的速度比较慢,如果能翻译成中文,那么以后再次查阅就可以节省很多时间。
    \item 三是当时国内大部分C++书籍都只到C++11,很少有这种讲述最新标准的书籍,就想着翻译之后分享给大家一起学习。
\end{itemize}

那个时候这本书还没有完成,我用了近一个月把当时已经有的所有章节翻译完,并以markdown格式
发布在了github上,随后就因为学业繁忙无暇再顾及这件事。

今年9月底时候意外发现这本书已经完成了,于是就想着翻译一下新增的章节。
但在阅读完整版的第一章时发现和原来相比有很多修改的地方,所以决定干脆重新翻译,
恰好当时又想学习一下\LaTeX,所以就决定用\LaTeX 重新翻译了。

在重新翻译的过程中,因为对\LaTeX 不是很熟悉,所以前前后后遇到了不计其数的问题。
这导致重新翻译的速度变慢了很多,持续了接近两个月。
为了达到接近于原版的排版效果,我把所有相关的\LaTeX 宏包的文档都查了个遍,
也在知乎、简书、stackoverflow等地方解决了很多疑难杂症。
这让我再一次相信学习\LaTeX 这种实践类知识的最佳方式就是实践。

在翻译的过程中我不仅学到了很多新的特性,而且还纠正了很多以前不清楚的地方
或者理解有偏差的地方。希望阅读本书的你也能有所收获。

\begin{flushright}
    ——MeouSker77\\
    2020.11.29
\end{flushright}


\section{致谢}
感谢batkiz(\url{https://github.com/batkiz})在去年和今年都帮忙校对了部分章节。

也感谢在github上star、fork、issue的读者,你们对本译文的关注也是我能坚持下去的动力之一。
